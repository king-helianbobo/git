\section{工具使用}
\subsection{shell}
\par 标准输入stdin,标准输出stdout,标准错误stderr,其文件描述符分别为0,1,2。>默认为将标准输出(stdout)重定向到其它地方,2>\&1表示将标准错误输出(stderr)重定向到标准输出(stdout)。\&>file表示把标准输出(stdout)和标准错误(stderr)都输出到文件file中,因此,2>1表示将stderr重定向到文件1中,而不是stdout。
\par let命令的替代表示形式是:((算术表达式))。例如,let 'j=i*6+2'等价于((j=i*6+2))。当表达式中有Shell的特殊字符时,必须用双引号或单引号将其括起来。例如,let ``val=a\textbar b''。如果不括起来,Shell会把命令行let val=a\textbar b中的''\textbar''看作管道符号,将其左右两边看成不同的命令,因此,将无法正确执行。 
\par linux中除了常见的读(r)、写(w)、执行(x)权限以外,还有3个特殊的权限,分别是setuid、setgid和stick bit。
\begin{verbatim}
[root@MyLinux ~]# ls -l /usr/bin/passwd /etc/passwd
-rw-r--r-- 1 root root  1549 08-19 13:54 /etc/passwd
-rwsr-xr-x 1 root root 22984 2007-01-07 /usr/bin/passwd
\end{verbatim}
\par /etc/passwd文件存放的各个用户的账号与密码信息,/usr/bin/passwd是执行修改和查看此文件的程序,但从权限上看,/etc/passwd仅有root权限的写(w)权,可实际上每个用户都可以通过/usr/bin/passwd命令去修改这个文件,于是这里就涉及了linux里的特殊权限setuid,正如-rwsr-xr-x中的s,setuid就是:让普通用户拥有可以执行“只有root权限才能执行”的特殊权限,setgid同理是让普通用户拥有“组用户”才能执行的特殊权限”。
\par /tmp目录是所有用户共有的临时文件夹,所有用户都拥有读写权限,这就必然出现一个问题,A用户在/tmp里创建了文件a,此时B用户看了不爽,在/tmp里把它给删了(因为拥有读写权限),那肯定是不行的。实际上不会发生这种情况,因为有特殊权限stick bit(粘贴位)权限,drwxrwxrwt中的最后一个t的意思是:除非目录的owner和root用户才有权限删除它,除此之外其它用户不能删除和修改这个目录。也就是说,/tmp目录中,只有文件的拥有者和root才能对其修改和删除,其他用户则不行,避免了上面所说的问题。
\par 如何设置以上特殊权限,如下所示,suid的二进制串为100,换算十进制为4,guid的二进制串为010,stick bit二进制串为001,换算成1。在一些文件设置了特殊权限后,字母不是小写的s或者t,而是大写的S和T,那代表此文件的特殊权限没有生效,是因为尚未赋予它对应用户的x权限。\\
\begin{minipage}{.5\linewidth}
\begin{verbatim}
setuid:chmod u+s xxx
setgid: chmod g+s xxx
stick bit : chmod o+t xxx
\end{verbatim}  
\end{minipage}
\begin{minipage}{.5\linewidth}
\begin{verbatim}
setuid:chmod 4755 xxx
setgid:chmod 2755 xxx
stick bit:chmod 1755 xxx
\end{verbatim}  
\end{minipage}
\subsection{git基本命令}
\par git add命令主要用于把要提交的文件的信息添加到索引库中,当我们使用\textbf{git commit}时,git将依据索引库中的内容来进行文件的提交。\textbf{git add <path>}表示add to index only files created or modified and not those deleted,因此\textbf{git add .}添加的文件不包括已经删除了的文件,<path>可以是文件也可以是目录。git不仅能判断出<path>中,修改(不包括已删除)了的文件,还能判断出新建的文件,并把它们的信息添加到索引库中。
\par \textbf{git add -u [<path>]}表示add to index only files modified or deleted and not those created,即把<path>中所有tracked文件中被修改过或已删除文件的信息添加到索引库,它不会处理untracked的文件。\textbf{git add -A [<path>]}表示把<path>中所有tracked文件中被修改过或已删除文件和所有untracked的文件信息添加到索引库。省略<path>表示'.',即当前目录。使用\textbf{git add .}后,如果打算撤销这次add,则使用命令:\textbf{git rm -r --cached ./},打算忽略整个目录时,可以添加.gitignore文件,表示忽略当前目录下的\textbf{mapreduce/ESMapReduce/lib/}目录。
\begin{verbatim}
*.class
# Package Files #
mapreduce/ESMapReduce/lib/
*.jar
*.war
*.ear
\end{verbatim}
\par git提交环节,存在三大部分:working tree,index file,commit。working tree是工作所在的目录,每当在代码中进行了修改,working tree的状态就改变了。index file是索引文件,它是连接working tree和commit的桥梁,每当我们使用git add命令后,index file的内容就改变了,此时index file和working tree完成了同步。commit是代码的一次提交,只有完成提交,代码才真正地进入git仓库,使用git commit就是将index file里的内容提交到commit中。因此,\textbf{git diff}是查看working tree与index file的差别的,\textbf{git diff --cached}是查看index file与commit的差别的,\textbf{git diff HEAD}是查看working tree和commit的差别的(注意,HEAD代表最近一次commit)。
\subsection{Maven使用初步}
\begin{enumerate}[(1)]
\item 配置maven环境,最主要的是设置环境变量:M2\_HOME,将其设置为maven安装目录,例子目录为:/usr/share/maven;
\item 修改仓库位置,仓库用于存放平时项目开发依赖的所有jar包。例子仓库路径:/opt/maven/repo,为设置仓库路径,必须修改\textbf{\${M2\_HOME}/conf}目录下的setting.xml文件。\\
<localRepository>/opt/maven/repo</localRepository> \\
在shell中输入并执行mvn help:system,如果没有错误,在仓库路径下应该多了些文件,这些文件是从maven的中央仓库下载到本地仓库的。
\item 创建maven项目,通过maven命令行方式创建一个项目,命令为:\textbf{mvn archetype:create -DgroupId=com.mvn.test -DartifactId=hello -DpackageName=com.mvn.test -Dversion=1.0}\\
由于第一次构建项目,所有依赖的jar包都要从maven的中央仓库下载,所以需要时间等待。做完这一步后,在工程根目录下应该有个pom.xml文件,其中groupId,artifactId和version比较常用。
\begin{itemize}
\item project:pom.xml文件的顶层元素;
\item modelVersion:指明POM使用的对象模型的版本,这个值很少改动。
\item groupId:指明创建项目的小组的唯一标识。GroupId是项目的关键标识,此标识以组织的完全限定名来定义。如org.apache.maven.plugins是所有Maven插件项目指定的groupId。
\item artifactId:指明此项目产生的主要产品的基本名称。项目的主要产品通常为一个jar包,源代码包通常使用artifactId作为最后名称的一部分。典型的产品名称使用这个格式:<artifactId>-<version>.<extension>(比如:myapp-1.0.jar)。
\item version:项目产品的版本号。Maven帮助你管理版本,可以经常看到SNAPSHOT这个版本,表明项目处于开发阶段。
\item name:项目的显示名称,通常用于maven产生的文档中。
\item url:指定项目站点,通常用于maven产生的文档中。
\item description:描述此项目,通常用于maven产生的文档中。
\end{itemize}
\item 项目hello已经创建完成,但它并不是eclipse所需要的项目目录格式,需要把它构建成eclipse可以导入的项目。进入到刚创建的项目目录(\textasciitilde/workspace/hello),执行:\textbf{mvn clean}(告诉maven清理输出目录target),然后执行\textbf{mvn compile}(告诉maven编译项目main部分的代码,或者两步合成一步\textbf{mvn clean compile}),此次仍会下载jar包到仓库中。编译后,项目的目录结构并不可以直接导入到Eclipse,需执行命令:\textbf{mvn eclipse:eclipse},命令执行完成后就能import到Eclipse。
\begin{figure}[htbp]
\centering\includegraphics[width=0.7\linewidth]{figures/maven-1.png}
\caption{Eclipse配置Maven仓库路径}\label{fig-maven-1}
\end{figure} 
\item 打开eclipse,在其中配置maven仓库路径,配置路径为:Window-->Perferences-->java-->Build Path-->Classpath Variables,新建变量(M2\_REPO)的类路径,如图\ref{fig-maven-1}示。
\item 包的更新与下载,如果发现junit版本比较旧,想换成新版本,修改项目下的的pom.xml文件为
\begin{verbatim}
<dependencies>
  <dependency>
     <groupId>junit</groupId>
     <artifactId>junit</artifactId>
     <version>4.8.1</version>
     <scope>test</scope>
  </dependency>
</dependencies>
\end{verbatim}
就可改变junit的版本号,在以后的maven操作中,maven会自动下载依赖的jar包。Maven中央仓库地址为http://search.maven.org,假如想下载struts的jar包,可在url内搜索struts。
\item 某些jar包可能位于远程机器上,因此需要配置maven仓库,配置仓库代码如下,其中的id属性无特别含义,仅用于标识仓库。当下载完某个jar包后,其在本地仓库的相对路径形如<groupId>/<artifactId>/<version>/*.jar,例如:/opt/maven/repo/org/ansj/tree\_split/1.0.1/tree\_split-1.0.1.jar。
\begin{verbatim}
<repositories>
  <repository>
    <id>ansj-maven-repo</id>
    <url>https://raw.github.com/ansjsun/mvn-repo/gh-pages</url>
  </repository>
  <repository>
    <id>remote-maven-repo</id>
    <url>http://search.maven.org</url>
  </repository>
</repositories>
\end{verbatim}
\item 有时候Maven访问很慢,可在settings.xml里重新设置中央仓库的镜像,中央仓库的默认地址为:http://repo.maven.apache.org/maven2/
\begin{verbatim}
<mirrors>
 <mirror>
   <id>ibiblio</id>
   <mirrorOf>central</mirrorOf>
   <name>Human Readable Name for this Mirror.</name>
   <url>http://mirrors.ibiblio.org/maven2</url>
 </mirror>
</mirrors>
\end{verbatim}
\end{enumerate}
\par 打算执行maven工程下某个类时,例如test目录下的某个类,执行命令如下:\textbf{mvn exec:java -X -Dexec.mainClass="org.ansj.liubo.test.test"  -Dexec.classpathScope=test},其中的classpathScope=test告诉maven打算执行test类,而非工程的main类。执行test部分的某些类之前,必须执行\textbf{mvn test-compile},以编译test部分的代码,如果没有,执行过程会报错。当准备向执行类添加参数时,使用如下命令。
\begin{verbatim}
mvn exec:java -Dexec.mainClass="org.ansj.liubo.test.test" 
  -Dexec.args="/mnt/f/tmp/content.txt /mnt/f/tmp/result3.txt"  
  -Dexec.classpathScope=test
\end{verbatim}
\subsection{Maven常用插件}
\par Maven本质上是一个插件框架,其核心并不执行任何具体的构建任务,所有这些任务都交给插件完成,例如编译源代码由maven-compiler-plugin完成。进一步说,每个任务对应一个插件,每个插件会有一个或者多个目标(goal),例如maven-compiler-plugin的compile目标用来编译位于src/main/java/目录下的主源码,而testCompile目标则用来编译位于src/test/java/目录下的code。
\par 用户可通过两种方式调用Maven插件目标。第一种方式是将插件目标与生命周期阶段(lifecycle phase)绑定,这样用户在命令行只输入了生命周期阶段,例如Maven默认将maven-compiler-plugin的compile目标与compile生命周期阶段绑定,因此命令\textbf{mvn compile}实际上先定位到compile这一生命周期阶段,然后再根据绑定关系调用maven-compiler-plugin的compile目标。第二种方式是直接在命令行指定要执行的插件目标,例如\textbf{mvn archetype:generate}就表示调用maven-archetype-plugin的generate目标,这种带冒号的调用方式与生命周期无关。
\par Maven有两个插件列表,第一个列表的GroupId为org.apache.maven.plugins,这里的插件最为成熟,具体地址为:http://maven.apache.org/plugins/index.html。第二个列表的GroupId为org.codehaus.mojo,这里的插件没有那么成熟,但也十分有用,地址为:http://mojo.codehaus.org/plugins.html。
\par 为使项目结构更为清晰,Maven区别对待Java代码文件和资源文件,maven-compiler-plugin用来编译java代码,maven-resources-plugin则用来处理resource文件。默认的资源文件目录是src/main/resources。很多用户会添加额外的资源文件目录,这个时候就可以通过配置maven-resources-plugin来实现。此外,资源文件过滤也是Maven的一大特性,可以在资源文件中使用\textbf{\${propertyName}}形式的Maven属性,然后配置maven-resources-plugin以开启对资源文件的过滤,之后就可以通过命令行或者Profile,针对不同环境传入不同的属性值,以实现灵活构建。
\par 由于历史原因,Maven2/3中用于执行测试的插件不是maven-test-plugin,而是maven-surefire-plugin。其实大部分时间内,只要测试类遵循通用的命令约定(以Test结尾、以TestCase结尾、或者以Test开头),就几乎不用知晓该插件是否存在。然而在当你想要跳过测试、排除某些测试类、或者使用一些Test特性时,了解maven-surefire-plugin的一些配置选项就很有用了。例如\textbf{mvn test -Dtest=FooTest}这样一条命令的效果是仅运行FooTest测试类,这是通过控制maven-surefire-plugin的test参数实现的。
\par Maven默认只允许指定一个主Java代码目录和一个测试Java代码目录,虽然这是一个应当尽量遵守的约定,但偶尔用户还是希望能够指定多个源码目录(例如为了应对遗留项目),build-helper-maven-plugin的add-source目标就服务于这个目的,通常它被绑定到默认生命周期的generate-sources阶段以添加额外的源码目录。这种做法是不推荐的,因为它破坏了Maven的约定,而且可能会遇到其他严格遵守约定的插件工具无法正确识别额外的source目录。build-helper-maven-plugin的另一个非常有用的目标是attach-artifact,使用该目标你可以以classifier的形式选取部分项目文件生成附属构件,并同时install到本地仓库,也可以deploy到远程仓库。
\par exec-maven-plugin很好理解,顾名思义,它能让你运行任何本地的系统程序,在某些特定情况下,运行一个Maven外部的程序可能就是最简单的问题解决方案,这就是exec:exec的用途,当然,该插件还允许你配置相关的程序运行参数。除了exec目标之外,exec-maven-plugin还提供了一个java目标,该目标要求你提供一个mainClass参数,然后它能够利用当前项目的依赖作为classpath,在同一个JVM中运行该mainClass。有时候,为了简单的演示一个命令行Java程序,可以在pom.xml中配置好exec-maven-plugin的相关运行参数,然后直接在命令运行\textbf{mvn exec:java}以查看运行效果。
\par 进行Web开发时,打开浏览器对应用进行手动的测试几乎是无法避免的,这种测试方法通常就是将项目打包成war文件,然后部署到Web容器中,再启动容器进行验证,这显然十分耗时。为了帮助开发者节省时间,jetty-maven-plugin应运而生,它完全兼容Maven项目的目录结构,能够周期性地检查源文件,一旦发现变更后自动更新到内置的Jetty Web容器中。做一些基本配置后(例如Web应用的contextPath和自动扫描变更的时间间隔),只要执行\textbf{mvn jetty:run},然后在IDE中修改代码,代码经IDE自动编译后产生变更,再由jetty-maven-plugin侦测到后将更新写入到Jetty容器,这时就可以直接测试Web页面。需要注意的是,jetty-maven-plugin并不是宿主于Apache或Codehaus的官方插件,因此使用的时候需要额外的配置settings.xml的pluginGroups元素,将org.mortbay.jetty这个pluginGroup加入。
\par 很多Maven用户遇到这样一个问题,当项目包含大量模块的时候,集体更新版本就变成一件烦人的事情,到底有没有自动化工具能帮助完成这件事情呢?(当然你可以使用sed之类的文本操作工具)答案是肯定的,versions-maven-plugin提供了很多目标帮助你管理Maven项目的各种版本信息。例如最常用的命令\textbf{mvn versions:set -DnewVersion=1.1-SNAPSHOT}就能帮助你把所有模块的版本更新到1.1-SNAPSHOT。该插件还提供了其他一些很有用的目标,display-dependency-updates能告诉你项目依赖有哪些可用的更新,类似的display-plugin-updates能告诉你可用的插件更新,use-latest-versions能自动帮你将所有依赖升级到最新版本。最后,如果对所做的更改满意,则可以使用\textbf{mvn versions:commit}提交,不满意的话也可以使用\textbf{mvn versions:revert}进行撤销操作。
