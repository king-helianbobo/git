%\usepackage{algorithm2e}
%\usepackage[linesnumbered]{algorithm2e}
%\usepackage[linesnumbered,ruled]{algorithm2e}
\usepackage{float} %%禁止浮动体四处浮动
\usepackage[linesnumbered,ruled]{algorithm2e}
\XeTeXlinebreaklocale "zh" %%以中文方式断行
\XeTeXlinebreakskip = 0pt plus 1pt %%断行间设置字符间距
\usepackage{fontspec}                
%% \setmainfont{宋体}
\usepackage{xeCJK}                   
\usepackage{verbatim}                   
%% \setCJKmainfont{WenQuanYi Zen Hei Sharp}
%% \setCJKmainfont[AutoFakeBold={true},AutoFakeSlant={true}]{宋体}
\setCJKmainfont[AutoFakeBold={true},AutoFakeSlant={true}]{KaiTi_GB2312}
\setmainfont{Consolas}
\setmonofont[AutoFakeBold={true},AutoFakeSlant={true},Scale=0.9]{Consolas}
%% \setCJKmonofont[AutoFakeBold={true},AutoFakeSlant={true},Scale=0.9]{KaiTi_GB2312}
\setCJKmonofont[AutoFakeBold={true},AutoFakeSlant={false},Scale=0.9]{KaiTi_GB2312}
%%\usepackage{indentfirst}             %% 首行缩进
\usepackage{enumerate}
\usepackage{array}
\usepackage{longtable}
\usepackage{tabularx,booktabs}
\usepackage{multicol}
%%\usepackage{fancyvrb}
\usepackage[nodisplayskipstretch]{setspace} \setstretch{1.3}
\usepackage{etoolbox}
\preto{\verbatim}{\color{blue}\edef\tempstretch{\baselinestretch}\par\setstretch{0.85}}
\appto{\endverbatim}{\vspace{-\tempstretch\baselineskip}\vspace{\baselineskip}\color{black}}


% \makeatletter
% \renewcommand{\verbatim@font}{\ttfamliy \small}
% \makeatother
%%%%%%%%%% 数学符号公式 %%%%%%%%%%

\usepackage{amsmath}                 % AMS LaTeX宏包
%\usepackage{amssymb}                 % 用来排版漂亮的数学公式
%\usepackage{amsbsy}
\usepackage{amsthm}
\usepackage{amsfonts}
\usepackage{mathrsfs}                % 英文花体字体
\usepackage{bm}                      % 数学公式中的黑斜体
\usepackage{bbding,manfnt}           % 一些图标,如 \dbend
\usepackage{lettrine}                % 首字下沉,命令\lettrine
\def\attention{\lettrine[lines=2,lraise=0,nindent=0em]{\large\textdbend\hspace{1mm}}{}}
%\usepackage{relsize}                 % 调整公式字体大小:\mathsmaller,\mathlarger
%\usepackage{caption2}                % 浮动图形和表格标题样式

%%%%%%%%%% 图形支持宏包 %%%%%%%%%%
\usepackage{graphicx}                % 嵌入png图像
\usepackage{color,xcolor}            % 支持彩色文本、底色、文本框等
%\usepackage{subfigure}
%\usepackage{epsfig}                 % 支持eps图像
%\usepackage{picinpar}               % 图表和文字混排宏包
%\usepackage[verbose]{wrapfig}       % 图表和文字混排宏包
%\usepackage{eso-pic}                % 向文档的部分页加n副图形, 可实现水印效果
%\usepackage{eepic}                  % 扩展的绘图支持
%\usepackage{curves}                 % 绘制复杂曲线
%\usepackage{texdraw}                % 增强的绘图工具
%\usepackage{treedoc}                % 树形图绘制
%\usepackage{pictex}                 % 可以画任意的图形

% \usepackage[dvipdfm,  %pdflatex,pdftex这里决定运行文件的方式不同
%             pdfstartview=FitH,
%             CJKbookmarks=true,
%             bookmarksnumbered=true,
%             bookmarksopen=true,
%             colorlinks, %注释掉此项则交叉引用为彩色边框(将colorlinks和pdfborder同时注释掉)
%             pdfborder=001,   %注释掉此项则交叉引用为彩色边框
%             linkcolor=green,
%             anchorcolor=green,
%             citecolor=green
%             ]{hyperref}  
\usepackage[colorlinks,
        % CJKbookmarks=true,
            linkcolor=red,
            anchorcolor=blue,
            citecolor=green
            ]{hyperref}%%引入标签

%%%%%%%%%% 粘贴源代码 %%%%%%%%%%
\usepackage{listings}                 % 粘贴源代码
\lstloadlanguages{R, C, csh, make}    % 所要粘贴代码的编程语言
\lstdefinelanguage{Renhanced}[]{R}{%
    morekeywords={acf,ar,arima,arima.sim,colMeans,colSums,is.na,is.null,%
    mapply,ms,na.rm,nlmin,replicate,row.names,rowMeans,rowSums,seasonal,%
    sys.time,system.time,ts.plot,which.max,which.min},
    deletekeywords={c},
    alsoletter={.\%},%
    alsoother={:_\$}}
\newcommand{\indexfonction}[1]{\index{#1@\texttt{#1}}}
\lstset{language=Renhanced,tabsize=4, keepspaces=true,
    xleftmargin=2em,xrightmargin=0em, aboveskip=1em,
    backgroundcolor=\color{gray!20},  % 定义背景颜色
    frame=none,                       % 表示不要边框
    extendedchars=false,              % 解决代码跨页时,章节标题,页眉等汉字不显示的问题
    basicstyle=\small,
    keywordstyle=\color{black}\bfseries,
    breakindent=10pt,
    identifierstyle=,                 % nothing happens
    commentstyle=\color{blue}\small,  % 注释的设置
    morecomment=[l][\color{blue}]{\#},
    numbers=left,stepnumber=1,numberstyle=\scriptsize,
    showstringspaces=false,
    showspaces=false,
    flexiblecolumns=true,
    breaklines=true, breakautoindent=true,breakindent=4em,
    escapeinside={/*@}{@*/},
}
%%\setlength{\parindent}{2.3em}         %%%%
\usepackage[top=1.1in,bottom=1.1in,left=1.2in,right=1.2in]{geometry}
%%\usepackage[top=1.1in,bottom=1.1in,left=0.2in,right=0.2in]{geometry}
\usepackage[sf]{titlesec}%% 控制标题的宏包
\usepackage{titletoc}%% 控制目录的宏包
\usepackage{caption2}%%浮动图形和表格标题样式
\usepackage{fancyhdr}%% fancyhdr宏包 页眉和页脚的相关定义
\usepackage{ifthen}
\usepackage{afterpage} %%保护脆弱命令
\usepackage{pifont}
\let\OldDing\ding
\renewcommand{\ding}[1]{\OldDing{#1}~}

