\newtheorem{example}{例}             % 整体编号
%% \newtheorem{algorithm}{算法}
\newtheorem{theorem}{定理}[section]  % 按section 编号
\newtheorem{definition}{定义}
\newtheorem{axiom}{公理}
\newtheorem{property}{性质}
\newtheorem{proposition}{命题}
\newtheorem{lemma}{引理}
\newtheorem{corollary}{推论}
\newtheorem{remark}{注解}
\newtheorem{condition}{条件}
\newtheorem{conclusion}{结论}
\newtheorem{assumption}{假设}

%%%%%%%%%% 一些重定义 %%%%%%%%%%
\renewcommand{\contentsname}{目录}     % 将Contents改为目录
\renewcommand{\abstractname}{摘要}     % 将Abstract改为摘要
\renewcommand{\refname}{参考文献}      % 将References改为参考文献
\renewcommand{\indexname}{索引}
\renewcommand{\figurename}{图}
\renewcommand{\tablename}{表}
\renewcommand{\appendixname}{附录}
\renewcommand{\proofname}{证明}
% \renewcommand{\algorithm}{算法}
%% add two ~ between \ref
\let\OldRef\ref
\renewcommand{\ref}[1]{~\OldRef{#1}~}



\newcommand{\yihao}{\fontsize{26pt}{36pt}\selectfont}    %% 一号, 1.4倍行距
\newcommand{\erhao}{\fontsize{22pt}{28pt}\selectfont}    %% 二号, 1.25倍行距
\newcommand{\xiaoer}{\fontsize{18pt}{18pt}\selectfont}   %%  小二, 单倍行距
\newcommand{\sanhao}{\fontsize{16pt}{24pt}\selectfont}   %% 三号, 1.5倍行距
\newcommand{\xiaosan}{\fontsize{15pt}{22pt}\selectfont}  %% 小三, 1.5倍行距
\newcommand{\sihao}{\fontsize{14pt}{21pt}\selectfont}    %% 四号, 1.5倍行距
\newcommand{\banxiaosi}{\fontsize{13pt}{19.5pt}\selectfont}    %% 半小四, 1.5倍行距
\newcommand{\xiaosi}{\fontsize{12pt}{18pt}\selectfont}    %% 小四, 1.5倍行距
\newcommand{\dawuhao}{\fontsize{11pt}{11pt}\selectfont}    %% 大五号, 单倍行距
\newcommand{\wuhao}{\fontsize{10.5pt}{10.5pt}\selectfont}    %% 五号, 单倍行距

%% 定义段落章节的标题和目录项的格式
\setcounter{secnumdepth}{5}
\setcounter{tocdepth}{3}
\titleformat{\section}[hang]{\ttfamily \bf \sanhao}{\mbox{\boldmath $\S$}\thesection}{0.5em}{}{}
\titlespacing{\section}{0pt}{3ex plus 0.5ex minus 0.5ex}{4.5ex plus .5ex minus .5ex}

\titleformat{\subsection}[hang]{\ttfamily \bf \xiaosan}{\mbox{\boldmath $\S$}\thesubsection}{0.5em}{}{}
\titlespacing{\subsection}{0pt}{1ex plus .2ex minus .2ex}{1ex plus .2ex minus .2ex}

\titleformat{\subsubsection}[hang]{\ttfamily \bf \sihao}{\mbox{\boldmath$\S$}\thesubsubsection }{0.5em}{}{}
\titlespacing{\subsubsection}{0pt}{0.8ex plus .2ex minus .2ex}{0.8ex plus .2ex minus .2ex}

\titleformat{\paragraph}[hang]{\ttfamily \bf \banxiaosi}{\mbox{\boldmath$\S$}\theparagraph}{0.5em}{}{}
\titlespacing{\paragraph}{0pt}{0.5ex plus .3ex minus .3ex}{0.5ex plus .3ex minus .3ex}


\titleformat{\subparagraph}[hang]{\ttfamily \bf \xiaosi }{\thesubparagraph}{0.5em}{}{}
\titlespacing{\subparagraph}{0pt}{0.5ex plus .3ex minus .3ex}{0.5ex plus .3ex minus .3ex}



%%  调整表格和插图的标号,不是全局显示,其计数器相对于当前节
\numberwithin{figure}{section}
\numberwithin{table}{section}
\renewcommand{\thefigure}{\thesection-\arabic{figure}}
\renewcommand{\thetable}{\thesection-\arabic{table}}
%% 调整表格和插图的标号,不是全局显示,其计数器相对于当前节,如图1-1,表2-1形状
\setlength{\parskip}{3pt plus1pt minus1pt} %段落之间的竖直距离
%% 定制浮动图形和表格标题样式
\renewcommand{\captionfont}{\small}
\renewcommand{\captionlabelfont}{\small}
\renewcommand{\captionlabeldelim}{\hspace{0.5em}}%%图表标签与标题的间距设为0.5em

\fancyhf{}\pagestyle{fancy}
\renewcommand{\sectionmark}[1]{\markboth{\thesection.#1}{}}
% \chead{\ifthenelse{
%     \isodd{\value{page}}}{\leftmark} {平时工作总结文档}}

% \chead{\ifthenelse{
%      \isodd{\value{page}}}{\leftmark} {\rightmark}}
\chead{\leftmark}
\fancyfoot[C]{\wuhao\thepage}

\newenvironment{myitemize}[1][]{%%%%%定义新环境
\begin{list}{{#1}} %%标签格式
    {
%%     \setlength{\leftmargin}{0cm}     %左边界
%%     \setlength{\parsep}{0ex}         %段落间距
%%     \setlength{\topsep}{0bp}         %列表到上下文的垂直距离
%%     \setlength{\itemsep}{0ex}        %标签间距
%%     \setlength{\labelsep}{0.3em}     %标号和列表项之间的距离,默认0.5emanation
     \setlength{\itemindent}{0pt}    %标签缩进量
%%     \setlength{\listparindent}{2em} %段落缩进量
    }}
{\end{list}}%%%%%
