\section{Lucene范围查询}
如果不需要范围查询,那么简单的将整数转换成字符串即可。但是要支持范围查询,又想提高效率,就十分麻烦。因为Lucene2.x/3.x索引的基本单位是Term,保存在tis或者tii(前者是字典文件,后者是字典的索引)文件里。lucene要求term都是按照字典序(lexicographic sortable)排列,然后它的范围查询根据tii找到范围的起始Term,然后把这中间的所有Term展开成一个BooleanQuery。
\par 如果简单的把数字变成字符串,那么2  4  8 31的字母顺序变成了 2 31 4 8,那么查找[2, 4]的RangeQuery也会把31也搜索出来,这显然是错误的。当然最容易想到的技巧就是在前面补0,比如把上面的数字变成字符串02 04 08 31,这样就不会有问题了。但是补多少个0是个问题。补0太多了,浪费空间(不过lucene的tis会使用前缀压缩,所以还不算太坏);补0太少了,不能保存太大的数值。
\par 这是RangeQuery的第一个问题,第二个问题就是展开成所有Term的Boolean查询有一个问题,那就是如果范围太大,那么可能包含非常多的Boolean Clause,较早的版本可能会抛出Too Many Boolean Clause的Exception。后来的版本做了改进,不展开所有的term,而是直接合并这些term的倒排表,这样的缺点是合并后的term的打分成了问题,比如tf,你是把所有term的tf加起来算一个term,idf呢,coord呢?算法暂且放下,即使可以合并成一个term,合并这些term的docIds也是很费时间,因为这些信息都在磁盘上。
\par Uwe Schindler(Generic XML-based Framework for Metadata Portals)基于Trie的数据结构做了优化。简单的介绍一下他的思路:
\par 首先可以把数值转换成一个字符串,并且保持顺序。如果$n_1 \le n_2$ ,那么$transform(n_1) \le transform(n_2)$。transform就是把数值转成字符串的函数,如果拿数学术语来说,transform是单调的。
\par 首先float可以转成int,double可以转成long,并且保持顺序。这个不难实现,因为float和int都是4个字节,double和long都是8个字节,从概念上讲,如果是用科学计数法,把指数放在前面就行了,因为指数大的肯定大,指数相同的尾数大的排前面。比如0.5e3,0.4e3,0.2e4,那么逻辑上保存的就是<4,0.2><3,0.5><3,0.4>,那么显然是保持顺序的。Java浮点数采用了ieee 754的表示方法,它的指数在前,尾数在后。这很好,不过有一点,它的最高位是符号位,正数0,负数1。这样就有点问题了。
\par 如果这个float是正数,那么把它看成int也是正数,而且根据前面的说明,指数在前,所以顺序也是保持好的。如果它是个负数,把它看作int也是负数,但是顺序就反了,举个例子 <4,-0.2> <3,-0.5>,如果不看符号,显然是前者大,但是加上符号,那么正好反过来。也就是说,负数的顺序需要反过来,怎么反过来呢?可以发现NumericUtils有这样一个方法,通过异或实现。
\begin{verbatim}
public static int floatToSortableInt(float val) {
   int f = Float.floatToIntBits(val);
   if (f<0) f ^= 0x7fffffff;
   return f;
}
\end{verbatim}
\par 一个int可以转换成一个字符串,并且保持顺序,这里考虑的是java的int,也就是有符号的32位整数,补码表示。如果只考虑正数,从0x0-0x7fffffff,那么它的二进制位是升序的(也就是把它看成无符号整数的时候);如果只考虑负数,从0x10000000-0xffffffff,那么它的二进制位也是升序的。唯一美中不足的就是负数排在正数后面。因此如果我们把正数的最高符号位变成1,把负数的最高符号位变成0,那么就可以把一个int变成有序的二进制位。可以在intToPrefixCoded看到这样的代码:
\begin{verbatim}
int sortableBits = val ^ 0x80000000;
\end{verbatim}
\par 因为lucene只能索引字符串,那么现在剩下的问题就是怎么把一个4个byte变成字符串了。Java在内存使用Unicode字符集,并且一个Java的char占用两个字节(16位),我们可能很自然的想到把4个byte变成两个char。但是Lucene保存Unicode时使用的是UTF-8编码,这种编码的特点是,0-127使用一个字节编码,大于127的字符一般两个字节,汉字则需要3个字节。这样4个byte最多需要6个字节。其实我们可以把32位的int看出5个7位的整数,这样的utf8编码就只有5个字节了。这段代码就是上面算法的实现:
