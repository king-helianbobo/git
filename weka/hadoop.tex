\section{关于MapReduce}
\par 下载\textsl{hadoop-1.2.1}包,然后使用\textsl{ant eclipse}命令,生成eclipse工程文件,在ant编译时,可能出现错误:\textsl{libtoolize: command not found},这是因为没有安装libtool包,使用\textsf{sudo apt-get install libtool}安装它们,这样,应该能生成工程文件。
\subsection{InputFormat类}
\par InputFormat类是Hadoop Map Reduce框架中的基础类之一。该类主要用来定义两件事情:
\begin{enumerate}[(1)]
\item 数据分割(Data splits),生成列表List<InputSplit>;
\item 记录读取器(Record reader),负责给执行任务的MapTask提供一个个Key/Value对。
\end{enumerate}
\par WordCount中的TextInputFormat(TextInputFormat --> FileInputFormat --> InputFormat, --> 标识继承关系),其核心功能是提供了getSplits函数(即数据分割部分,实现过程在FileInputFormat类中)和LineRecordReader类。
\par 数据分割(InputSplit)是MapReduce框架中的基础概念之一,它定义了单个MapTask的输入数据源(MapTask可以没有输入数据源)的大小及其所处的DataNode的位置信息。WordCount例子中,仍然是按照块大小(默认情况下,64M)将文本文件分成以64M字节为单位的一个个InputSplit,每个InputSplit将作为分配任务的基本单元,即每个MapTask的数据输入源。
\par RecordReader主要负责从输入的hdfs文件(逻辑split,也就是每个MapTask执行的输入数据源)上读取数据并将它们以键值对的形式提交给MapTask(一次Map任务的执行过程称为MapTask,封装于独立的Java Virtual Machine中),RecordReader提供抽象的nextKeyValue方法,用于判断输入InputSplit中是否还有待处理的Key/Value对。
\par WordCount示例中,RecordReader的具体实现类为LineRecordReader,对任意一个InputSplit,它读取的Start位置并不是该DataBlock的开头,而是跳转到第一行单词的末尾,以第一行的末尾作为读取的Start位置。因此,LineRecordReader,在很大可能性上不只读取当前的DataBlock,它还会读取InputSplit的下个DataBlock,直到获得一个完整的字符串行(单词),这样就能避免一行单词被两个InputSplit拆分。
\subsection{OutputFormat类}
OutputFormat类是Hadoop Map Reduce框架中的基础类之一。该类主要用来定义两件事情:
\begin{enumerate}[(1)]
\item 提供RecordWriter,MapReduce作业的输出均由RecordWriter负责,它负责写入一个个Key/Value对,至于写到hdfs还是本地文件系统,或者写入一个还是多个文件,取决于RecordWriter的实现。
\item 提供OutputCommitter,OutputCommitter负责:作业的初始化(在MapReduce中称为JobSetupTask),作业失败时的清除(JobCleanupTask),任务失败时的清除(TaskCleanupTask)和作业提交(commitJob),它的默认实现是FileOutputCommitter。
\end{enumerate}
\par OutputCommiter执行的任务包括:
\begin{enumerate}[(1)]
\item JobSetupTask,作业开始时,在hdfs上建立作业临时目录,调用OutputCommitter的setupJob接口。
\item TaskCleanupTask,构成作业的某个Task失败时,调用OutputCommitter的abortTask接口。
\item JobCleanupTask,作业结束时,如果作业失败,调用OutputCommitter的abortJob接口,如果作业成功,调用OutputCommitter的commitJob接口。
\end{enumerate}
\par MapTask和ReduceTask都可能会调用RecordWriter,这个输出会写到哪里去呢?如果不存在Reduce任务,那么Map任务的输出当然是由RecordWriter负责写入了。查阅MapTask类的源码,有如下代码:
\begin{verbatim}
if (job.getNumReduceTasks() == 0) {
  output = new NewDirectOutputCollector;
} else {
  output = new NewOutputCollector;
}
\end{verbatim}
\par 也就是说,如果没有ReduceTask,则RecordWriter对MapTask负责,反之,RecordWriter对ReduceTask负责,NewOutputCollector将负责MapTask的输出,它调用MapOutputBuffer来输出其中间结果,这种情况下,中间结果将写入本地文件系统,避免了写入HDFS带来的较大开销,但这种情况下,中间结果对用户是不可见的。
\subsection{Two Mappers}
\par MapReduce中存在两种API,旧的API来自于包org.apache.hadoop.mapred,新的API来自于包org.apache.hadoop.mapreduce。新的API有Mapper<KEYIN, VALUEIN, KEYOUT, VALUEOUT>类,调用者通过继承Mapper类,并重载void map(KEYIN key, VALUEIN value, Context context) 函数,以执行map任务。
\par Context类继承了MapContext<KEYIN,VALUEIN,KEYOUT,VALUEOUT>,是Mapper中核心类,给MapTask,ReduceTask提供包括:
\begin{itemize}
\item InputFormat,包含了RecordReader<KEYIN,VALUEIN>,InputSplit等信息,其中reader负责给Mapper提供K/V对,InputSplit定义了MapTask的输入数据源;
\item OutputFormat,包含了OutputCommitter,RecordWriter<KEYOUT,VALUEOUT>等信息,writer负责写入Mapper或Reducer的计算结果;
\item StatusReporter,JobID,TaskAttemptID等信息,还包括OutputCommitter,committer负责Task的一些清理工作,可以更多地与MapReduce执行交互;
\item 调用不同的RecordReader,Old API调用getRecordReader(InputSplit split, JobConf job, Reporter reporter),New API调用createRecordReader(InputSplit split, TaskAttemptContext context)函数。
\item 调用不同的RecordWriter,旧API调用getRecordWriter(FileSystem ignored, JobConf job, String name, Progressable progress)函数,新API调用getRecordWriter(TaskAttemptContext context)函数。
\end{itemize}
\subsection{配置log4j}
开发MapReduce作业时,使用System.out.print()方式和将日志输出到本地文件中比较麻烦,因此使用log4j库打印日志。但使用Hadoop库时,自己写的log4j.xml(或log4j.properties)文件会被MapReduce框架覆盖掉,从而不起作用,此时只能使用Hadoop提供的log4j.properties文件。在自己写的类的main()函数中,可以显式地加载log4j.xml或log4j.properties文件,代码为:\textbf{DOMConfigurator.configure("src/log4j.xml");}或者\textbf{PropertyConfigurator.configure("src/log4j.properties");}
\par 修改\$\{HADOOP\_HOME\}/etc/hadoop/目录下的log4j.properties文件,在文件末尾添加如下几句:
\begin{verbatim}
log4j.logger.org.jpgExtractor=INFO,jpgExtractor
log4j.appender.jpgExtractor=org.apache.log4j.FileAppender
log4j.appender.jpgExtractor.File=/tmp/log4j.log
log4j.appender.jpgExtractor.layout=org.apache.log4j.PatternLayout
log4j.appender.jpgExtractor.layout.ConversionPattern=%d %-5p [%c{1}] %m%n
log4j.additivity.org.jpgExtractor=false
# log4j.appender.jpgExtractor.filter=org.apache.log4j.varia.LevelRangeFilter
# log4j.appender.jpgExtractor.filter.LevelMin=DEBUG
# log4j.appender.jpgExtractor.filter.LevelMax=ERROR
\end{verbatim}
\par 这表示将package(org.jpgExtractor)下的Log输出到名为jpgExtractor的LogAppender中。Namenode(JobTracker)和Datanode(TaskTracker)下的log4j.properties文件都必须按如上所说进行修改,修改完后重启整个Hadoop。
\par 当一个作业执行时,由Namenode(JobTracker)执行的那部分代码(如InputFormat中的getSplits函数和isSplitable函数),其输出结果保存在"/tmp/log4j.log"文件中,而由Datanode(TaskTracker)执行的那部分代码,其输出结果没有放在"/tmp/log4j.log"文件里,而是输出到了\$\{HADOOP\_HOME\}/logs/userlogs/\$\{jobId\}/\$\{TaskAttemptId\}/stdlog文件中。
\par 修改了的配置文件只对JobTracker,TaskTracker等MapReduce框架中的代码有效,而对执行MapTask,ReduceTask的Java虚拟机无效,因为当一个Map或Reduce任务执行时,MapReduce框架重新给运行任务的JVM设置了参数,这些参数不再调用Hadoop配置目录下的log4j.properties了。
\par 执行任务的JVM参数被重新设置,至于Hadoop是怎么设置的,查阅Hadoop源码中的log4j.properties配置文件,发现TLA这个\textsl{TaskLogAppender},其描述如下:
\begin{verbatim}
log4j.appender.TLA=org.apache.hadoop.mapred.TaskLogAppender
log4j.appender.TLA.taskId=${hadoop.tasklog.taskid}
log4j.appender.TLA.isCleanup=${hadoop.tasklog.iscleanup}
log4j.appender.TLA.totalLogFileSize=${hadoop.tasklog.totalLogFileSize}
log4j.appender.TLA.layout=org.apache.log4j.PatternLayout
log4j.appender.TLA.layout.ConversionPattern=%d{ISO8601} %p %c: %m%n
\end{verbatim}
然后通过\textbf{find ./ -iname '*.java' |xargs grep -n 'TLA'}命令,找到如下代码:\textbf{vargs.add("-Dhadoop.root.logger=INFO,TLA");}
\par MapReduce框架在开启执行MapTask或ReduceTask的JVM时,修改了\textsl{hadoop.root.logger},从而将rootLogger的输出重定向到syslog文件中。为了实现在Mapper和Reducer中输出Log信息到本地文件系统或其它地方时,继承Mapper或Reducer的setup(context)函数,在其中添加如下代码:
\begin{verbatim}
protected void setup(Context context) 
      throws IOException,InterruptedException{
  FileAppender  fa = new FileAppender();
  fa.setName("FileLogger");
  fa.setFile("/tmp/lau.log");
  fa.setLayout(new PatternLayout("%d %-5p [%c{1}] %m%n"));
  fa.setThreshold(Level.INFO);
  fa.setAppend(true);
  fa.activateOptions();
  Logger.getRootLogger().getLoggerRepository().resetConfiguration();
  Logger.getRootLogger().addAppender(fa);
}
\end{verbatim}
\par 这种方式可以输出用户作业的日志到本地文件("/tmp/lau.log"),但对OutputCommiter的日志输出无效,因为如上代码只能修改Mapper和Reducer子进程的setup函数,而无法修改执行SetupTask和CleanupTask的Java虚拟机的默认设置,下面附上Log4j中ConversionPattern参数的格式意义:
\begin{verbatim}
%c 输出日志信息所属类的全名
%d 输出日志时间点,默认格式为ISO8601,也可以指定格式,比如:%d{yyyy-MM-dd HH:mm:ss},输出类似:2013-10-18 22:10:28
%l 输出日志事件的发生位置,包括类名、发生的线程,以及在代码中的行数
形如org.soul.library.UserDefineLib.loadFile(UserDefineLib.java:165)
%L 输出日志事件在代码中的行数。
%m 输出代码中指定的信息,如log.info(message)中的message;
%n 输出一个回车换行符,Windows平台为'\r\n',Unix平台为'\n';
%p 输出优先级,即DEBUG,INFO,WARN,ERROR,FATAL。如果是调用debug()输出的,则为DEBUG,依此类推
%r 输出自应用启动到输出该日志信息所耗费的毫秒数
%t 输出产生该日志事件的线程名
\end{verbatim}
\subsection{Job的提交过程}
在org.apache.hadoop.mapreduce.Job类中,submit函数将作业提交给MapReduce框架执行,其由waitForCompletion函数调用。
Submit函数 --> setUseNewAPI(设置使用新的API函数) --> 生成JobClient对象(JobClient创建与JobTracker的RPC协议)
现在需要搞清楚的是Job的提交过程,
\par 当在hadoop集群上运行作业时,不能用Ctrl+C来杀掉作业,可行的办法是执行:hadoop job -list,查看Hadoop集群上有哪些作业,然后调用:hadoop job -kill 'jobId',来杀掉标识符为jobId的作业。
\subsection{Java客户端访问hdfs}
\par Java客户端指的是,在非hadoop集群上的机器访问hdfs。
\begin{enumerate}[(1)]
\item HDSF添加用户并设置权限,Hadoop为每个用户设置一定的权限,当使用Java Client端访问hdfs时,实际上是以某个用户来访问hdfs,hdfs会检查该用户是否具有相应权限,比如hadoop的root用户具有对全部文件的读写权限。在hdfs上建立用户,并赋予用户权限的过程如下:
\begin{verbatim}
hadoop fs -mkdir /user (如果没有/user目录)
hadoop fs -mkdir /user/liubo(建立liubo目录)
hadoop fs -chown liubo:liubo /user/liubo
\end{verbatim}
\par 默认情况下的读写权限为700,也就是说,只有user具备读写权限(hdfs无可执行文件,因此没有必要设置执行权限,700实际上是600)
\item Java客户端访问hdfs时,读取用户名过程为:先读取HADOOP\_USER\_NAME系统环境变量,如果为null,则读取java环境变量中的HADOOP\_USER\_NAME,如果仍然为null,则以系统当前的user为默认的用户名。比如在Ubuntu 12.04下的的用户名是lau,那么,当没有设置系统环境变量和java环境变量时,Java Client会以lau这个用户名访问hdfs,假如lau这个用户没有权限,则执行会出错。
\item 设置系统环境变量比较麻烦,更改系统当前登录用户更加麻烦,最容易的方式是设置Java环境变量,设置代码为:
\begin{verbatim}
Properties property = System.getProperties();
property.setProperty("HADOOP_USER_NAME", "root");
\end{verbatim}
\par 以这种方式,就可以读写hdfs目录下的任何文件了。从这里也能看出,hdfs的安全机制十分薄弱,必须依赖于宿主操作系统的安全机制。
\end{enumerate}
\section{Hadoop-2.4.0安装}
\subsection{前言}
\par 本文目的是为当前最新版本的Hadoop 2.4.0提供详细的安装说明,以帮助减少安装困难,并对一些错误原因进行说明。安装只涉及了hadoop-common、hadoop-hdfs、hadoop-mapreduce与hadoop-yarn,并不包含其它技术。
\par 共3台机器,部署如下所示,在shell中可通过hostname Slave1修改IP地址为71的主机名。
\begin{itemize}
\item 192.168.50.70(Master),角色为用于hdfs的NameNode,SecondaryNameNode以及用于Yarn框架的ResourceManager与NodeManager.
\item 192.168.50.71(Slave1),角色为DataNode,NodeManager.
\item 192.168.50.72(Slave2),角色为DataNode,NodeManager.
\end{itemize}
\subsection{安装Java与Hadoop}
\begin{enumerate}[(1)]
\item 下载jdk-Version.tar.gz(本例中jdk版本是1.7.0\_60),解压缩jdk至/home/java目录;
\item 在/root/.bash\_profile文件末尾写入:
\begin{verbatim}
JAVA_HOME=/home/java/jdk1.7.0_60
PATH=$PATH:$JAVA_HOME/bin
CLASSPATH=.:$JAVA_HOME/lib/dt.jar:$JAVA_HOME/lib/tools.jar
export JAVA_HOME PATH CLASSPATH
\end{verbatim}
\item 保存文件并执行(source /root/.bash\_profile)使配置生效,执行java -version,如果java版本正确,则安装成功。
\end{enumerate}
\par Hadoop安装与配置类似于jdk,解压缩后配置下环境变量即可,执行hadoop version,如果版本符合期望值,安装成功。
\subsection{配置ssh}
\par 由于开启与关闭Hadoop均在Master上执行,为方便与Slaves的交互,需要配置Master无密码登录所有的Slave节点。
\begin{enumerate}[(1)]
\item 在Master机器上生成密码对(ssh-keygen),采用默认路径(/root/.ssh/id\_rsa)保存。
\item 将id\_rsa.pub写入到信任key列表中,注意中间符号为\textbf{>>},而非\textbf{>}.
\begin{verbatim}
## 实际上是对localhost免密码登录
cat /root/.ssh/id_rsa.pub >> /root/.ssh/authorized_keys
\end{verbatim}
\item 修改文件"authorized\_keys"权限,使得只有root具备读写权限。
\begin{verbatim}
chmod 600 /root/.ssh/authorized_keys
\end{verbatim}
\item 配置ssh,在/etc/ssh/sshd\_config文件中写入如下内容,并重启Dameon进程(service sshd restart)。
\begin{verbatim}
RSAAuthentication yes # 启用 RSA 认证
PubkeyAuthentication yes # 启用公钥私钥配对认证方式
AuthorizedKeysFile .ssh/authorized_keys # 公钥文件路径
\end{verbatim}
\item 把Master公钥复制到所有的Slave机器,以192.168.50.71为例,执行如下步骤:
\begin{verbatim}
## Master上执行:scp root/.ssh/id_rsa.pub root@192.168.50.71:/tmp
## 50.71上执行:cat /tmp/id_rsa.pub >> /root/.ssh/authorized_keys
## 50.71上执行:chmod 600 /root/.ssh/authorized_keys
\end{verbatim}
\end{enumerate}
\par 至此已经实现了Matser到Slave1的无密码登陆,Matser无密码登陆Slave2可重复上面的步骤。Slave无密码登录Master和Slave间相互无密码登录原理相同,重要的是把一方机器的公钥追加到另一机器的"/root/.ssh/authorized\_keys"中.
\subsection{集群配置}
\begin{enumerate}[(1)]
\item 每台机器上执行命令,确保每台机器时间同步(所谓同步,按照我的理解,任意两台机器的时间差控制在一定范围内),否则MapReduce作业运行时会出现Token超时等错误。
\begin{verbatim}
# 与时间服务器同步
ntpdate time.nist.gov
# 将本地时间从EDT(美国东部时间)变为CST
mv /etc/localtime /etc/localtime.bak
ln -s /usr/share/zoneinfo/Asia/Shanghai /etc/localtime
\end{verbatim}
\item 修改etc/hadoop/slaves文件
\begin{verbatim}
localhost
Slave1
Slave2
\end{verbatim}
\item 修改core-site.xml,hdfs-site.xml,mapred-sit.xml
\begin{table}[h]
  \centering
  \begin{tabular}{|l|l|}
\hline
fs.defaultFS &hdfs://Master:9001 \\ \hline
hadoop.tmp.dir & /home/hadoop-2.4.0/tmp \\ \hline
dfs.namenode.rpc-address & Master:9001\\ \hline
dfs.namenode.secondary.http-address& Master:50090 \\ \hline
dfs.namenode.name.dir   & /home/hadoop-2.4.0/name\\ \hline
dfs.namenode.data.dir   &   /home/hadoop-2.4.0/data\\ \hline
mapreduce.framework.name &  yarn \\ \hline
  \end{tabular}
\end{table}
\par 上表中的主机名也可换成对应IP,如192.168.50.70:9001\footnote{建议配置成IP方式,方便局域网内查看。},fs.defaultFS中的端口地址必须和dfs.namenode.rpc-address相同,如果没有配置dfs.namenode.rpc-address,则启动时会报错,而且必须同时指定IP和端口。
\item 启动运行,首先格式化NameNode,然后分别启动hdfs和yarn。
\begin{verbatim}
hdfs namenode -format
./sbin/start-dfs.sh
./sbin/start-yarn.sh
\end{verbatim}
\item 如果在启动出现如下错误,可以在hadoop-env.sh 加入下面两句话,使用JDK提供的jps命令\footnote{简单察看当前机器上java进程的一些情况。},查看相应的进程是否启动。
\begin{verbatim}
export HADOOP_COMMON_LIB_NATIVE_DIR=${HADOOP_HOME}/lib/native
export HADOOP_OPTS="-Djava.library.path=$HADOOP_HOME/lib"
\end{verbatim}
\item 发布hadoop jar包,编译完后的tar.gz包在hadoop-dist/target/目录下。
\begin{verbatim}
## 不带natice code
$ mvn package -Pdist -DskipTests -Dtar
## 带natice code,推荐方式
$ mvn package -Pdist,native -DskipTests -Dtar
\end{verbatim}
\end{enumerate}
\subsection{hadoop编译打包}
\par 从网上下载hadoop-2.4.0-src.tar.gz,解压之,然后执行\textbf{mvn clean install -DskipTests}命令。可能遇到的错误是protoc包没有安装,执行如下几个命令安装protoc。
\begin{verbatim}
wget https://protobuf.googlecode.com/files/protobuf-2.5.0.tar.gz
./configure --prefix=/usr
make
make check
make install
protoc --version
\end{verbatim}
如果提示libprotoc版本仍然是2.4.1,而hadoop要求的版本为2.5.0,mvn编译将不能通过,此时需要配置环境变量。
\begin{verbatim}
emacs /etc/profile
export PROTOC_HOME=/opt/protobuf-2.5.0
export PATH=$PATH:$PROTOC_HOME/src
\end{verbatim}
\par 使用命令\textbf{mvn package -Pdist -DskipTests -Dtar}生成可运行的hadoop二进制包,如果没有错误,生成的tar.gz文件应在./hadoop-dist/target目录下。
\par 如果想导入Eclipse,则进入hadoop-maven-plugins目录,执行命令\textbf{mvn install}。返回basedir目录,执行命令\textbf{mvn eclipse:eclipse -DskipTests}生成Eclipse Project描述文件。


