\section{关于MapReduce}
\par 下载\textsl{hadoop-1.2.1}包,然后使用\textsl{ant eclipse}命令,生成eclipse工程文件,在ant编译时,可能出现错误:\textsl{libtoolize: command not found},这是因为没有安装libtool包,使用\textsf{sudo apt-get install libtool}安装它们,这样,应该能生成工程文件。
\subsection{InputFormat类}
\par InputFormat类是Hadoop Map Reduce框架中的基础类之一。该类主要用来定义两件事情:
\begin{enumerate}[(1)]
\item 数据分割(Data splits),生成列表List<InputSplit>;
\item 记录读取器(Record reader),负责给执行任务的MapTask提供一个个Key/Value对。
\end{enumerate}
\par WordCount中的TextInputFormat(TextInputFormat --> FileInputFormat --> InputFormat, --> 标识继承关系),其核心功能是提供了getSplits函数(即数据分割部分,实现过程在FileInputFormat类中)和LineRecordReader类。
\par 数据分割(InputSplit)是MapReduce框架中的基础概念之一,它定义了单个MapTask的输入数据源(MapTask可以没有输入数据源)的大小及其所处的DataNode的位置信息。WordCount例子中,仍然是按照块大小(默认情况下,64M)将文本文件分成以64M字节为单位的一个个InputSplit,每个InputSplit将作为分配任务的基本单元,即每个MapTask的数据输入源。
\par RecordReader主要负责从输入的hdfs文件(逻辑split,也就是每个MapTask执行的输入数据源)上读取数据并将它们以键值对的形式提交给MapTask(一次Map任务的执行过程称为MapTask,封装于独立的Java Virtual Machine中),RecordReader提供抽象的nextKeyValue方法,用于判断输入InputSplit中是否还有待处理的Key/Value对。
\par WordCount示例中,RecordReader的具体实现类为LineRecordReader,对任意一个InputSplit,它读取的Start位置并不是该DataBlock的开头,而是跳转到第一行单词的末尾,以第一行的末尾作为读取的Start位置。因此,LineRecordReader,在很大可能性上不只读取当前的DataBlock,它还会读取InputSplit的下个DataBlock,直到获得一个完整的字符串行(单词),这样就能避免一行单词被两个InputSplit拆分。
\subsection{OutputFormat类}
OutputFormat类是Hadoop Map Reduce框架中的基础类之一。该类主要用来定义两件事情:
\begin{enumerate}[(1)]
\item 提供RecordWriter,MapReduce作业的输出均由RecordWriter负责,它负责写入一个个Key/Value对,至于写到hdfs还是本地文件系统,或者写入一个还是多个文件,取决于RecordWriter的实现。
\item 提供OutputCommitter,OutputCommitter负责:作业的初始化(在MapReduce中称为JobSetupTask),作业失败时的清除(JobCleanupTask),任务失败时的清除(TaskCleanupTask)和作业提交(commitJob),它的默认实现是FileOutputCommitter。
\end{enumerate}
\par OutputCommiter执行的任务包括:
\begin{enumerate}[(1)]
\item JobSetupTask,作业开始时,在hdfs上建立作业临时目录,调用OutputCommitter的setupJob接口。
\item TaskCleanupTask,构成作业的某个Task失败时,调用OutputCommitter的abortTask接口。
\item JobCleanupTask,作业结束时,如果作业失败,调用OutputCommitter的abortJob接口,如果作业成功,调用OutputCommitter的commitJob接口。
\end{enumerate}
\par MapTask和ReduceTask都可能会调用RecordWriter,这个输出会写到哪里去呢?如果不存在Reduce任务,那么Map任务的输出当然是由RecordWriter负责写入了。查阅MapTask类的源码,有如下代码:
\begin{verbatim}
if (job.getNumReduceTasks() == 0) {
  output = new NewDirectOutputCollector;
} else {
  output = new NewOutputCollector;
}
\end{verbatim}
\par 也就是说,如果没有ReduceTask,则RecordWriter对MapTask负责,反之,RecordWriter对ReduceTask负责,NewOutputCollector将负责MapTask的输出,它调用MapOutputBuffer来输出其中间结果,这种情况下,中间结果将写入本地文件系统,避免了写入HDFS带来的较大开销,但这种情况下,中间结果对用户是不可见的。
\subsection{Mapreduce中的两个Mapper}
\par MapReduce中存在两种API,旧的API来自于包org.apache.hadoop.mapred,新的API来自于包org.apache.hadoop.mapreduce。新的API有Mapper<KEYIN, VALUEIN, KEYOUT, VALUEOUT>类,调用者通过继承Mapper类,并重载void map(KEYIN key, VALUEIN value, Context context) 函数,以执行map任务。
\par Context类继承了MapContext<KEYIN,VALUEIN,KEYOUT,VALUEOUT>,是Mapper中核心类,给MapTask,ReduceTask提供包括:
\begin{itemize}
\item InputFormat,包含了RecordReader<KEYIN,VALUEIN>,InputSplit等信息,其中reader负责给Mapper提供K/V对,InputSplit定义了MapTask的输入数据源;
\item OutputFormat,包含了OutputCommitter,RecordWriter<KEYOUT,VALUEOUT>等信息,writer负责写入Mapper或Reducer的计算结果;
\item StatusReporter,JobID,TaskAttemptID等信息,还包括OutputCommitter,committer负责Task的一些清理工作,可以更多地与MapReduce执行交互;
\item 新API调用\textbf{inputFormat.createRecordReader(split,taskContext);}旧API调用\textbf{job.getInputFormat().getRecordReader(split,job,reporter)}函数。
\end{itemize}
\subsection{Hadoop中配置log4j}
开发MapReduce作业时,使用System.out.print()方式和将日志输出到本地文件中比较麻烦,因此使用log4j库打印日志。但使用Hadoop库时,自己写的log4j.xml(或log4j.properties)文件会被MapReduce框架覆盖掉,从而不起作用,此时只能使用Hadoop提供的log4j.properties文件。在自己写的类的main()函数中,可以显式地加载log4j.xml或log4j.properties文件,代码为:\textbf{DOMConfigurator.configure("src/log4j.xml");}或者\textbf{PropertyConfigurator.configure("src/log4j.properties");}
\par 修改\$\{HADOOP\_HOME\}/etc/hadoop/目录下的log4j.properties文件,在文件末尾添加如下几句:
\begin{verbatim}
log4j.logger.org.jpgExtractor=INFO,jpgExtractor
log4j.appender.jpgExtractor=org.apache.log4j.FileAppender
log4j.appender.jpgExtractor.File=/tmp/log4j.log
log4j.appender.jpgExtractor.layout=org.apache.log4j.PatternLayout
log4j.appender.jpgExtractor.layout.ConversionPattern=%d %-5p [%c{1}] %m%n
log4j.additivity.org.jpgExtractor=false
# log4j.appender.jpgExtractor.filter=org.apache.log4j.varia.LevelRangeFilter
# log4j.appender.jpgExtractor.filter.LevelMin=DEBUG
# log4j.appender.jpgExtractor.filter.LevelMax=ERROR
\end{verbatim}
\par 这表示将package(org.jpgExtractor)下的Log输出到名为jpgExtractor的LogAppender中。Namenode(JobTracker)和Datanode(TaskTracker)下的log4j.properties文件都必须按如上所说进行修改,修改完后重启整个Hadoop。
\par 当一个作业执行时,由Namenode(JobTracker)执行的那部分代码(如InputFormat中的getSplits函数和isSplitable函数),其输出结果保存在"/tmp/log4j.log"文件中,而由Datanode(TaskTracker)执行的那部分代码,其输出结果没有放在"/tmp/log4j.log"文件里,而是输出到了\$\{HADOOP\_HOME\}/logs/userlogs/\$\{jobId\}/\$\{TaskAttemptId\}/stdlog文件中。
\par 修改了的配置文件只对JobTracker,TaskTracker等MapReduce框架中的代码有效,而对执行MapTask,ReduceTask的Java虚拟机无效,因为当一个Map或Reduce任务执行时,MapReduce框架重新给运行任务的JVM设置了参数,这些参数不再调用Hadoop配置目录下的log4j.properties了。
\par 执行任务的JVM参数被重新设置,至于Hadoop是怎么设置的,查阅Hadoop源码中的log4j.properties配置文件,发现TLA这个\textsl{TaskLogAppender},其描述如下:
\begin{verbatim}
log4j.appender.TLA=org.apache.hadoop.mapred.TaskLogAppender
log4j.appender.TLA.taskId=${hadoop.tasklog.taskid}
log4j.appender.TLA.isCleanup=${hadoop.tasklog.iscleanup}
log4j.appender.TLA.totalLogFileSize=${hadoop.tasklog.totalLogFileSize}
log4j.appender.TLA.layout=org.apache.log4j.PatternLayout
log4j.appender.TLA.layout.ConversionPattern=%d{ISO8601} %p %c: %m%n
\end{verbatim}
然后通过\textbf{find ./ -iname '*.java' |xargs grep -n 'TLA'}命令,找到如下代码:\textbf{vargs.add("-Dhadoop.root.logger=INFO,TLA");}
\par MapReduce框架在开启执行MapTask或ReduceTask的JVM时,修改了\textsl{hadoop.root.logger},从而将rootLogger的输出重定向到syslog文件中。为了实现在Mapper和Reducer中输出Log信息到本地文件系统或其它地方时,继承Mapper或Reducer的setup(context)函数,在其中添加如下代码:
\begin{verbatim}
protected void setup(Context context) 
      throws IOException,InterruptedException{
  FileAppender  fa = new FileAppender();
  fa.setName("FileLogger");
  fa.setFile("/tmp/lau.log");
  fa.setLayout(new PatternLayout("%d %-5p [%c{1}] %m%n"));
  fa.setThreshold(Level.INFO);
  fa.setAppend(true);
  fa.activateOptions();
  Logger.getRootLogger().getLoggerRepository().resetConfiguration();
  Logger.getRootLogger().addAppender(fa);
}
\end{verbatim}
\par 这种方式可以输出用户作业的日志到本地文件("/tmp/lau.log"),但对OutputCommiter的日志输出无效,因为如上代码只能修改Mapper和Reducer子进程的setup函数,而无法修改执行SetupTask和CleanupTask的Java虚拟机的默认设置,下面附上Log4j中ConversionPattern参数的格式意义:
\begin{verbatim}
%c 输出日志信息所属类的全名
%d 输出日志时间点,默认格式为ISO8601,也可以指定格式,比如:%d{yyyy-MM-dd HH:mm:ss},输出类似:2013-10-18 22:10:28
%l 输出日志事件的发生位置,包括类名、发生的线程,以及在代码中的行数,形如org.ansj.library.UserDefineLibrary.loadFile(UserDefineLibrary.java:165)。
%L 输出日志事件在代码中的行数。
%m 输出代码中指定的信息,如log.info(message)中的message;
%n 输出一个回车换行符,Windows平台为'\r\n',Unix平台为'\n';
%p 输出优先级,即DEBUG,INFO,WARN,ERROR,FATAL。如果是调用debug()输出的,则为DEBUG,依此类推
%r 输出自应用启动到输出该日志信息所耗费的毫秒数
%t 输出产生该日志事件的线程名
\end{verbatim}
\subsection{Job的提交过程}
在org.apache.hadoop.mapreduce.Job类中,submit函数将作业提交给MapReduce框架执行,其由waitForCompletion函数调用。
Submit函数 --> setUseNewAPI(设置使用新的API函数) --> 生成JobClient对象(JobClient创建与JobTracker的RPC协议)
现在需要搞清楚的是Job的提交过程,
\par 当在hadoop集群上运行作业时,不能用Ctrl+C来杀掉作业,可行的办法是执行:hadoop job -list,查看Hadoop集群上有哪些作业,然后调用:hadoop job -kill 'jobId',来杀掉标识符为jobId的作业。
\subsection{Java客户端访问hdfs}
\par Java客户端指的是,在非hadoop集群上的机器访问hdfs。
\begin{enumerate}[(1)]
\item HDSF添加用户并设置权限,Hadoop为每个用户设置一定的权限,当使用Java Client端访问hdfs时,实际上是以某个用户来访问hdfs,hdfs会检查该用户是否具有相应权限,比如hadoop的root用户具有对全部文件的读写权限。在hdfs上建立用户,并赋予用户权限的过程如下:
\begin{verbatim}
hadoop fs -mkdir /user (如果没有/user目录)
hadoop fs -mkdir /user/liubo(建立liubo目录)
hadoop fs -chown liubo:liubo /user/liubo
\end{verbatim}
\par 默认情况下的读写权限为700,也就是说,只有user具备读写权限(hdfs无可执行文件,因此没有必要设置执行权限,700实际上是600)
\item Java客户端访问hdfs时,读取用户名过程为:先读取HADOOP\_USER\_NAME系统环境变量,如果为null,则读取java环境变量中的HADOOP\_USER\_NAME,如果仍然为null,则以系统当前的user为默认的用户名。比如在Ubuntu 12.04下的的用户名是lau,那么,当没有设置系统环境变量和java环境变量时,Java Client会以lau这个用户名访问hdfs,假如lau这个用户没有权限,则执行会出错。
\item 设置系统环境变量比较麻烦,更改系统当前登录用户更加麻烦,最容易的方式是设置Java环境变量,设置代码为:
\begin{verbatim}
Properties property = System.getProperties();
property.setProperty("HADOOP_USER_NAME", "root");
\end{verbatim}
\par 以这种方式,就可以读写hdfs目录下的任何文件了。从这里也能看出,hdfs的安全机制十分薄弱,必须依赖于宿主操作系统的安全机制。
\end{enumerate}