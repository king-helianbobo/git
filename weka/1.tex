\subsection{ElasticSearch源码分析}
\par elasticSearch对Google Guice进行了简单的封装,通过ModulesBuilder类构建es的模块,一个es节点包括下面模块:
\begin{multicols}{2}
\begin{itemize}
\item PluginsModule:插件模块
\item SettingsModule:参数设置模块
\item NodeModule:节点管理模块
\item NetworkModule:网络管理模块
\item NodeCacheModule:分布式缓存模块
\item ScriptModule:脚本处理模块
\item JmxModule:Java管理扩展模块
\item EnvironmentModule:环境模块
\item NodeEnvironmentModule:节点环境模块
\item ClusterNameModule:集群名模块
\item ThreadPoolModule:线程池模块
\item DiscoveryModule:节点自动发现模块
\item ClusterModule:集群模块
\item RestModule:rest模块
\item TransportModule:tcp模块
\item HttpServerModule:http模块
\item RiversModule:river模块
\item IndicesModule:索引模块
\item SearchModule:搜索模块
\item ActionModule:行为模块
\item MonitorModule:监控模块
\item GatewayModule:持久化模块
\item NodeClientModule:客户端模块
\end{itemize}
\end{multicols}
\begin{enumerate}[(1)]
\item 有"/\_analyze"和"/\{index\}/\_analyze两种,接受GET和POST方法,在org/elasticsearch/rest/action/admin/indices/analyze/RestAnalyzeAction.java中被处理;在ActionModule中,存在一个注册函数registerAction(AnalyzeAction.INSTANCE, TransportAnalyzeAction.class),因此请求将由TransportAnalyzeAction类处理。
\item "/\{index\}/\_close",只接受POST方法,在org/elasticsearch/rest/action/admin/indices/close/RestCloseIndexAction.java中被处理;
\item  创建一个新的索引,json参数中包括settings和mappings(\textbf{注意并非endpoint中的\_mapping}),只有"/\{index\}"一种操作,接受POST和PUT两种方法,在org/elasticsearch/rest/action/admin/indices/create/RestCreateIndexAction.java中处理;
\item rest.action目录负责接受,解释Rest请求。client.node负责管理节点客户端请求。action.admin.indices目录负责处理Rest请求中的动作。
\end{enumerate}