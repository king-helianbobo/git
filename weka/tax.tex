\section{地税数据}
\par 搜索地税数据时,会遇到Excel表中小数四舍五入的情况,如搜索身份证号码:320219195609196990
\par 对整数的处理,当用户输入整数时,例如“10”,则扩展成“[10,10.0]”,前者用于查询文本域和整数域,后者只用来查询小数域。
\par 对日期的处理,当用户输入日期以查询时,索引中存在两部分日期:(1)日期域(2)文本域,文本中有一部分日期信息,如“2014年1月份开票项目”。当输入“2008年9月”,此时可以扩展成两部分:“[2008,年,9,月]”和“[2008年09月]”,前者用于查询文本域,后者用来查询日期域。
\par 当用户输入查询字符串时,是不是需要返回每个token的类型呢?如输入“10”,扩展后为“[10,10.0]”,类型分别为“[integer,float]”。当用户输入“20080909”时,扩展后为“[2008-09-09,20080909]”,类型分别为“[date,integer]”。
