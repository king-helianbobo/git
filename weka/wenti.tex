\section{实施方案}
\begin{enumerate}[(1)]
\item 繁简转换,(问题不难,使用繁简转换表)。
\item 拼音转汉字,(属于优化的问题,现在没必要做)。
\item 同音词拼写错误,先把汉字转成拼音,然后进行可能的比对,比如中华人名共和国-->中华人民共和国,这个已经实现。
\item 英文拼写错误,(暂时不作考虑,可以考虑使用开源的库)。
\item 形近词错误,(这个需要吗,当用户使用五笔或记错某个字的样子时)
\end{enumerate}
\subsection{拼写错误检查}
\par 拼音转汉字想法是较为直接的,建立一个以拼音为term的查询词索引,posting list中只保存查询频率最高的K个查询词,如
\begin{table}[h]
  \centering
  \begin{tabular}{|l|l|}
\hline
jiujingkaoyan &“久经考验”,“酒精考验”\\ \hline
zhijiucaotang & “子九草堂', “子久草堂” \\ \hline
xufuniuza & “徐福牛杂”,“许府牛杂”,“徐府牛杂”\\ \hline
shoujichongzhi&“手机冲值”, “手机充值”\\ \hline
  \end{tabular}
\end{table}
\par 这一步可以在自动提示中使用,但自动提示与它的区别是,自动提示在拼音输入了一部分的情况下也要提示,比如输入 “xufu”就要提示“许府牛杂”。
\par 同音词拼写错误也基于同样的想法,但是需要一个可能出错的查询词列表,这个列表可以为借鉴于下列几种情况:
\begin{enumerate}[(1)]
\item 以carot为例,返回有carot的文档,也返回一些包含纠错后的term carrot和torot的文档。
\item 与(1)相似,但仅当carot不在词典中时,返回纠错后的结果。
\item 与(1)相似,但仅当包含carot的文档数小于一个预定义的阈值时,即当原始查询返回文档数小于预定义的阈值时,搜索引擎给出纠错后的词列表。
\end{enumerate}
\par 情况(1)相当于是对所有查询都进行纠错处理,发现那些搜索比较少的,就给出一个纠错提示,比如“天浴”搜索次数比较少,而“天娱”搜索次数比较多,那么在用户搜索“天浴”时就提示“天娱”,即使“天浴”也是一个正常的查询词。情况(2)就是当查询没有获得文档,才对它进行纠错处理,然后查询相应结果。情况(3)是一个查询它返回的文档数少于一个预定义阈值时,才进行纠错处理。
\par 英文拼写错误,在lucene中已有贡献者实现了spellchecker模块,主要算法有:Jaro Winkler distance,Levenstein Distance(Edit Distance),NGram Distance。但Lucene中的实现过于简单,使用两两比较,时间复杂性是$O(n^2)$。
\par 形近字错误,形近字一般是用户记错了形声字,或是使用五笔的用户输入错误。在网上可以下载SunWb\_mb文件,它里面包含五笔的编码和笔画的编码,但字根比如“马”比“口”笔画更多,也更有代表性,但在这种方法中却是相同的。
\par 方言纠错,可以用soudex进行纠错
\subsection{网站群的一些问题}
\par 政府网站中的特殊字符需要进行转换。
\par 当对政府网站中的网页建立索引时,如果使用url作为文档id,相同内容的文档可能出现多次。下面两个url仅仅是字符大小写不同,却作为不同的id出现。解决办法是文档id一律小写。
\begin{verbatim}
http://www.wuxi.gov.cn/WebPortal/AskAnswer/Gov_AskAnswer_Info?
AnswerID=9174e3c9-4bb6-41cd-ba75-6465a3aa490c
http://www.wuxi.gov.cn/WEBPORTAL/AskAnswer/Gov_AskAnswer_Info?
AnswerID=9174e3c9-4bb6-41cd-ba75-6465a3aa490c
\end{verbatim}
\par 下面两个url均包含了“$\backslash$”字符,其后跟随普通ASCII字符时,转化成一种转义字符,作为文档id会出错。
\begin{verbatim}
http://www.wuxi.gov.cn/WEBPORTAL/ChiefHall/ChiefHallInfoDetailsXK?SystemID
=4028818a28aff36d0128b3861a090c76&ChiefHallType=xk360ChromeURL\Shell\Open\Command
http://www.wuxi.gov.cn/WEBPORTAL/ChiefHall/ChiefHallInfoDetailsXK?SystemID
=4028818a2fbfee4b012fc2fe529909ac&ChiefHallType=xk360ChromeURL\Shell\Open\Command
\end{verbatim}
\par 爬虫获得的网页内容中,如标题为《京杭大运河无锡段》,而内容为《京杭大运河无锡段 发布时间: 2011年11月23日 修改时间: 2013年10月22日 [ 大 中 小 ] 浏览次数:》,这部分内容重复,影响了检索的相关性,需将其过滤。
\par “春涛”这个词,结合上下文,分词结果不同,下面两个例子,前者分成了“春,涛”,后者分成了“春涛”,还有词,比如:“易视腾”,“买卖宝”。
\begin{verbatim}
curl -XGET 'namenode:9200/official_mini/_analyze?analyzer=soul_index&pretty' -d '登太湖仙岛、观鼋渚春涛'
curl -XGET 'namenode:9200/official_mini/_analyze?analyzer=soul_index&pretty' -d '鼓浪屿上听春涛'
\end{verbatim}
\par 对2014年3月18号爬取的网站群数据,作分词处理后,共有164976个不同的词,总共的词个数是161630251个,使用word2vec代码如下,最后一行代码对SogouR.txt进行编码转换。
\begin{verbatim}
./word2vec -train /mnt/f/b.txt -output vectors.bin -cbow 0 -size 200 -window 7 -negative 0 -hs 1 -sample 1e-3 -threads 12 -binary 1
./distance vectors.bin
cat SogouR.txt | iconv -f gbk -t utf8 -c > SogouR-utf8.txt
\end{verbatim}
\par synonym-new.txt中与synonym.txt没有交集的词条共5689个,与其有包含或被包含关系的词条共1464个(有效条目是1318个),剩下的都是有交集的,扩展后的词条为12715个。synonym.txt中与synonym-new.txt没有交集的词条共1893个,最终的词条应该是21613个。
\par 在ShardSuggestService中,有若干个隶属于IndexShard的cache,目前保存的cache主要有:用于拼写检查的spellCheckCache,用于智能提示的titleSuggestCache,用于统计每个term\footnote{Lucene术语,代表一个Token。}的document frequency(term的文档数目)和total frequency(term总共出现次数)的termListCache。使用如下命令,获取term“垃圾”的两个频率,当结合词性识别关键字时,该命令非常有用(命令中的size暂时没用)。
\begin{verbatim}
curl -XGET 'localhost:9200/official_mini/_termlist?pretty' -d '{
    "action": "termlist", 
    "fields": [
        "content", 
        "contenttitle"
    ], 
    "size": 0, 
    "term": "垃圾"
}'
\end{verbatim}
\par 使用如下命令,获取“驾驶证补办”的标题提示。
\begin{verbatim}
curl -X POST localhost:9200/official_mini/table/__suggest?pretty -d '{
    "field": "contenttitle", 
    "size": 15, 
    "term": "驾驶证补办", 
    "type": "synonym"
}'
\end{verbatim}
\par 第一个命令,装载标题域《contenttitle》的cache,如果cache存在,则直接返回,否则装载之。第二个命令,刷新标题域《contenttitle》的cache,如果cache不存在,则返回,否则刷新该cache。type为suggest表示装载titleSuggestCache,如果为spell,表示装载spellCheckCache。
\begin{verbatim}
curl -XPOST localhost:9200/official_mini/table/__suggestRefresh -d'{
    "action": "load", 
    "field": "contenttitle", 
    "type": "suggest"
}'
curl -XPOST localhost:9200/official_mini/table/__suggestRefresh -d'{   
    "action": "refresh", 
    "field": "contenttitle", 
    "type": "suggest"
}'
\end{verbatim}
\par 使用curl命令搜索第7页(注意pn=60,百度默认一个页面10条记录)的“成奎安”,
\begin{verbatim}
http://www.baidu.com/s?wd=成奎安&pn=60
先将cookies存入文件cookie.txt(位于当前目录)
curl -c cookie.txt www.baidu.com
cat cookie.txt
当前url使用指定的cookies文件,-A后的字符串为user agent,该agent拷贝自FireFox。
curl -v --cookie ./cookie.txt  -A "Mozilla/4.0 (compatible; MSIE 8.0; Windows NT 6.1)"  "http://www.baidu.com/s?^&wd=成奎安&pn=0";
《url为http://www.baidu.com/s?cl=3^&wd=×××》貌似也可。
如被百度防御系统封闭,考虑删除cookies文件,再重新生成cookies文件。
\end{verbatim}
